\documentclass[12pt]{beamer}
\usepackage[utf8]{inputenc}
\usecolortheme{default}
% \setbeamercovered{transparent}


\newcommand{\prule}{\vspace{-8pt}\rule{50pt}{0.5pt}}




\title[Modale Prädikatenlogik]{Philosophische Modale Prädikatenlogik}
\subtitle{Eine sehr kurze Einführung}
\author{Conrad Friedrich}
\institute[Uni Köln]{Universität zu Köln}
\date{\today}



\begin{document}

\begin{frame}
\titlepage
\end{frame}

\section{Prädikatenlogik}

\begin{frame}
  \Huge Prädikatenlogik \\
  \pause \large ``Baby Logic" Version
\end{frame}

\begin{frame}{Formalisierung}
  \begin{columns}

    \begin{column}[t]{5cm}
      \begin{enumerate}
      \item<1-> Alle Menschen sind sterblich \\
      \item<1-> Sokrates ist ein Mann\\
      \item<2-> Also: Sokrates ist sterblich
      \end{enumerate}
    \end{column}

    \begin{column}[t]{5cm}
      \begin{enumerate}
      \item<3-> $P$
      \item<3-> $Q$
      \item<3-> Also: $R$
      \end{enumerate}
    \end{column}

  \end{columns}

  \vspace{10 mm}
  \begin{itemize}
  \item<4-> Mehr Struktur, als wir mit der Aussagenlogik abbilden
    können
  \end{itemize}
\end{frame}

\begin{frame}{Stattdessen: Formalisierung II}
  
  \begin{columns}

    \begin{column}[t]{5cm}
      \begin{enumerate}
      \item<1-> Alle Menschen sind sterblich.
      \item<1-> Sokrates ist ein Mensch.
      \item<1-> Also: Sokrates ist sterblich.
      \end{enumerate}
    \end{column}

    \begin{column}[t]{5cm}
      \begin{enumerate}
      \item<2-> $\forall x (Mx \to Sx)$.
      \item<3-> $Ms$.
      \item<4-> Also: $Ss$.
      \end{enumerate}
    \end{column}

  \end{columns}

\end{frame}

\begin{frame}{Formalisierung III}
  \begin{itemize}
  \item<1-> Es gibt genau einen Gott.
  \item<2-> gdw. Es gibt ein Ding, das Gott ist, und alle anderen
    Dinge sind, falls sie Gott sind, identisch mit diesem Ding.
  \item<3-> gdw. $\exists x(Gx \land \forall y(Gx \to x=y))$.
  \end{itemize}
\end{frame}

\begin{frame}{Vokabular}
  \begin{itemize}
  \item<1-> Variablen
    \begin{itemize}
    \item $\forall \mathbf{x} (M\mathbf{x} \to S\mathbf{x})$
    \end{itemize}
  \item<2-> Konstanten
    \begin{itemize}
    \item $P\mathbf{c}$
    \end{itemize}
  \item<3-> Prädikatensymbole (n-stellig)
    \begin{itemize}
    \item $\forall x (\mathbf{M}x \to \mathbf{S}x)$, $\mathbf{G}xy$
    \end{itemize}
  \item<4-> Konnektive wie in der Aussagenlogik
    \begin{itemize}
    \item $\neg, \to, \land, \lor, \leftrightarrow$
    \end{itemize}
  \item<5-> Quantorsymbole
    \begin{itemize}
    \item $\forall, \exists$
    \end{itemize}
  \item<6-> (Funktionssymbole, Hilfszeichen...)
  \end{itemize}
\end{frame}


\begin{frame}{Grammatik}
  \begin{enumerate}
  \item<1-> Wenn $t_1,...,t_n$ Variablen oder Konstanten sind und $P$
    ein $n$-stelliges Prädikat ist, dann ist $Pt_1,...,t_n$ eine
    (atomare, wohlgeformte) Formel.
    \begin{itemize}
    \item<2-> $Fx$, $Rab$
    \end{itemize}
  \item<3-> Wenn $A$ und $B$ Formeln sind, dann sind auch $\neg A$,
    $A\to B$, $A \land B$, $A \lor B$, $A \leftrightarrow B$ Formeln.
  \item<4-> Wenn A eine Formel ist und $x$ eine Variable, dann sind
    $\forall xA$ und $\exists x A$ Formeln.
  \end{enumerate}
\end{frame}

\begin{frame}{Grammatik II: Beispiele}
  Seien $a,b,c$ Konstanten und $P,Q$ Prädikatensymbole. Was wird
  intuitiv mit diesen Formeln ausgedrückt?

  \begin{itemize}
  \item<2-> $Pa$, $Qab$
  \item<3-> $\neg Pa$
  \item<4-> $Qaa$
  \item<5-> $Qab \leftrightarrow Qba$
  \item<6-> $(Qab \land Qbc) \to Qac$
  \end{itemize}

\end{frame}

\begin{frame}{Grammatik III: Beispiele}
  Seien $x, y, z$ Variablen, $a,b,c$ Konstanten und $P,Q$
  Prädikatensymbole. Was wird intuitiv mit diesen Formeln ausgedrückt?

  \begin{itemize}
  \item<2-> $Px$, $Qxy$
  \item<3-> $\exists x Px$
  \item<4-> $\forall x (Px \to Qax)$
  \item<5-> $\exists y Qxy$
  \item<6-> $\forall x \exists y Qxy$
  \end{itemize}

  \begin{itemize}
  \item<7-> Variablen, die in einer Formel im Skopus eines Quantors
    stehen, sind \emph{gebunden}, sonst \emph{frei}.
  \item<8-> Formeln, in denen keine freien Variablen vorkommen, heißen
    \emph{Sätze}.
  \end{itemize}
\end{frame}

\begin{frame}{Semantik}
  \begin{itemize}[<+->]
  \item Wann würden wir $Qab$ intuitiv als wahr bezeichnen?

  \item Wenn die Dinge, für die $a,b$ stehen, tatsächlich die
    Eigenschaft haben, die mit $Q$ bezeichnet wird.
  \item $Px$?

  \item Die Variable $x$ hat die Eigenschaft $P$?
  \item Nonsense. Wir können nur \emph{Sätzen} Wahrheitswerte zuordnen
    (zumindest ohne Weiteres.)
  \end{itemize}
\end{frame}

\begin{frame}{Semantik II}
  \begin{itemize}[<+->]
  \item Wann würden wir intuitiv $\exists x Px$ als wahr bezeichnen?
  \item Wenn \emph{irgendein} Ding $P$ erfüllt.
  \item $\forall x Fx$?
  \item Wenn \emph{alle} Dinge $F$ erfüllen.
  \end{itemize}
\end{frame}

\begin{frame}{Semantik III - Interpretation}
  Eine Interpretation $I$ besteht aus einem Tupel
  $\langle D,v\rangle$.
  \begin{itemize}[<+->]
  \item $D$ ist der (nicht-leere) Gegenstandsbereich, über die
    quantifiziert wird.
  
  \item $v$ ist eine Funktion, so dass
    \begin{itemize}
    \item Wenn $c$ eine Konstante ist, dann ist $v(c) \in D$.
    \item Wenn $P$ ein $n$-stelliges Prädikatensymbol ist, dann ist
      $v(P) \subseteq D^n$.
    \end{itemize}
  \item (Tafel)
  \end{itemize}

\end{frame}

\begin{frame}{Semantik IV: Wahrheit in einer Interpretation}
  \begin{itemize}[<+->]
  \item $v(Pc_1,...c_n) = 1$ gdw.
    $\langle v(c_1),...,v(c_n)\rangle \in v(P)$, sonst 0.
    \begin{itemize}
    \item $Fa$ ist wahr gdw.\\ $v(a)$ (das Objekt von $a$) in $v(F)$
      (der Extension von $F$) enthalten ist
    \end{itemize}
  \item $v(\neg A)$, $v(A \land B)$ usw. genau wie in der
    Aussagenlogik.
  \end{itemize}

  \begin{itemize}[<+->]
    
  \item $v(\forall xA) = 1$ gdw. \textbf{jedes} Objekt des
    Gegenstandsbereiches $A$ erfüllt, sonst 0.
  \item $v(\exists xA) = 1$ gdw. \textbf{mindestens ein} Objekt des
    Gegenstandsbereiches $A$ erfüllt, sonst 0.
    \begin{itemize}
    \item Eine Formel $A$ gilt in $I$ gdw. $v(I) = 1$.
    \item (Formal unterbestimmt. Was heißt `erfüllen'?)
    \end{itemize}
    % \item $v(\forall xA) = 1$ gdw. \textbf{für alle}
    %   $\partial \in D$
    %   gilt: $v(A_x(c_\partial)) = 1$, sonst 0.
    % \item $v(\exists xA) = 1$ gdw. \textbf{für mindestens ein}
    %   $\partial \in D$ gilt: $v(A_x(c_\partial)) = 1$, sonst 0.

  \end{itemize}

\end{frame}

% \begin{frame}{Semantik V: Wahrheit in einer Interpretation II}
%   \begin{block}{Notation} Sei $\partial \in D$ und $c_\partial$ eine
%     Konstante, sodass $v(c_\partial) = \partial$.  Dann ist
%     $A_x(c_\partial)$ die Formel, die wir erhalten, wenn wir alle
%     Vorkommnisse von $x$ in $A$ durch $c_\partial$ ersetzen.
%     \\
%     \pause
%     \begin{itemize}
%     \item Beispiel: $A$ sei $Px \land Qy$. $A_x(c_\partial)$ ist
%       dann $Pc_\partial \land Qy$.
%     \end{itemize}
%   \end{block}
%   \pause

% \end{frame}

\begin{frame}{Semantik VI: Interpretation Beispiel}

  Konstanten: $a, b, c$. Prädikatensymbole: $P, Q$. \\
  Sei $I$ gegeben durch: $D = \{\partial_a, \partial_b, \partial_c\}$,\\
  $v(a) = \partial_a$ usw., \\
  $v(P) = \{\partial_a, \partial_b\}$, $v(Q) = \{\langle \partial_a, \partial_a \rangle\, \langle \partial_c, \partial_b \rangle \}$.\\
  Welche der folgenden Formeln gilt in $I$?

  \pause

  \begin{itemize}[<+->]
  \item $Pa \lor Qac$.
  \item $\exists x (Qxx \land Px)$.
  \item $\forall x (Px \to \exists y Qxy)$.
  \item $\forall x (Px \lor Qxy).$
  \end{itemize}

\end{frame}

\begin{frame}{Tableaux-Regeln}

\begin{block}{Notation}
  $A_x(c)$ ist die Formel, die wir erhalten, wenn wir alle freien
  Vorkommnisse von $x$ in $A$ durch $c$ ersetzen.
\end{block}
\pause
\begin{columns}

  \begin{column}[t]{.25 \linewidth}
    \begin{center}
      $\forall x A$ \\
      \prule\\
      $A_x(a)$
    \end{center}
    $a$ ist eine Konstante, die schon vorkam.
  \end{column}
  \pause
  \begin{column}[t]{.25 \linewidth}
    \begin{center}
      $\exists x A$ \\
      \prule\\
      $A_x(c)$
    \end{center}
    $c$ ist eine neue Konstante.
  \end{column}
  \pause
  \begin{column}[t]{.25 \linewidth}
    \begin{center}
      $\neg \exists x A$\\
      \prule\\
      $\forall x \neg A$\\
    \end{center}
  \end{column}
  \pause
  \begin{column}[t]{.25 \linewidth}
    \begin{center}
      $\neg \forall x A$ \\
      \prule\\
      $\exists x \neg A$
    \end{center}
  \end{column}
\end{columns}

\vspace{1em}
\begin{columns}

  \pause

  \begin{column}[t]{.5 \linewidth}
    \begin{itemize}[<+->]
    \item $Pc \vdash \exists x Px$?
    \end{itemize}
  \end{column}

  \begin{column}[t]{.5 \linewidth}
    \begin{itemize}[<+->]
    \item $\forall x \neg Px \vdash \neg \exists x Px$?
    \item $\exists x \neg Px \vdash \neg \forall x Px$?
    \end{itemize}

  \end{column}
\end{columns}

\end{frame}

\begin{frame}
  \Huge Modale Prädikatenlogik\\
  \large Constant Domain
\end{frame}

\begin{frame}{Vokabular und Grammatik}
  \begin{block}{Vokabular}
    Zum Vokabular werden $\Box$ und $\Diamond$ hinzugefügt.
  \end{block}

  \pause
  \begin{block}{Grammatik}
    Wir erweitern die Grammatik der Prädikatenlogik, so dass: \pause
    \begin{itemize}[<+->]
    \item Wenn $A$ eine Formel ist, dann ist auch $\Box A$ eine
      Formel.
    \item Wenn $A$ eine Formel ist, dann ist auch $\Diamond A$ eine
      Formel.
      \begin{itemize}
      \item $\forall x \Box (Px \land Qx) \to \Box \forall x Px$
      \item
        $\Box \Diamond \exists x Px \to \Box \exists x \Diamond (Px
        \lor Qx)$
      \end{itemize}
    \item Restliche Grammatik wie in der klassischen Prädikatenlogik.
    \end{itemize}
  \end{block}

\end{frame}

\begin{frame}{Semantik: Intuitiv}
  Wir wollen die Logik so erweitern, dass: \pause
  \begin{itemize}[<+->]
  \item Ein Gegenstand in einer Situation eine Eigenschaft haben kann,
    aber an einer anderen Welt nicht.
  \item $v(Pa) = 1$ an $w_0$, aber $v(Pb) = 0$ an $w_1$.
  \end{itemize}
\end{frame}

\begin{frame}{Semantik: Interpretationen}
  Die Interpretation $I$ wird erweitert, so dass sie aus einem 4-tupel
  $\langle D, W, R, v \rangle$ besteht, wobei \pause
  \begin{itemize}[<+->]
  \item $D$ wie in der klassischen Prädikatenlogik der nicht-leere
    Gegenstandsbereich ist,
  \item $W$ wie in der modalen Aussagenlogik eine Menge möglicher
    Welten ist,
  \item $R$ wie in der modalen Aussagenlogik eine Relation auf $W$ ist
    (d.h. $R \subseteq W \times W$),
  \item $v$ eine Funktion ist, so dass
    \begin{itemize}
    \item Wenn $c$ eine Konstante ist, dann ist $v(c) \in D$.
    \item Wenn $P$ ein $n$-stelliges Prädikatensymbol ist und
      $w \in W$, dann ist $v_w(P) \subseteq D^n$.
    \end{itemize}
  \end{itemize}
\end{frame}

\begin{frame}{Semantik II}
  Eine Formel ist nun wahr oder falsch in einer Interpretation
  \emph{an einer Welt}.  Also
  \begin{itemize}[<+->]
  \item $v_\mathbf{w}(Pc_1,...c_n) = 1$ gdw.
    $\langle v(c_1),...,v(c_n)\rangle \in v_\mathbf{w}(P)$, sonst 0.
  \item Ganz analog mit nicht atomaren Formeln:
    \begin{itemize}
    \item $v_w(\neg A) = 1$ gdw. $v_w(A) = 0$.
    \item $v_w(A \land B) = 1$ gdw. $v_w(A) = 1$ und $v_w(B) = 1$,
      sonst 0.
    \item usw.
    \end{itemize}
  \item $v_w(\Box A) = 1$ gdw. \textbf{für alle} $w' \in W$ mit $wRw'$
    gilt: $v_{w'}(A) = 1$.
  \item $v_w(\Diamond A) = 1$ gdw. \textbf{für mindestens ein}
    $w' \in W$ mit $wRw'$ gilt: $v_{w'}(A) = 1$.
  \end{itemize}
\end{frame}

\begin{frame}{Semantik: Beispiel}
  Sei eine Interpretation $I$ gegeben durch: \pause
  \begin{itemize}[<+->]
  \item $D = \{\partial_a, \partial_b, \partial_c\}$
  \item $W = \{w_0, w_1\}$
  \item
    $R = \{\langle w_0, w_0\rangle ,\langle w_0, w_1 \rangle , \langle
    w_1, w_1 \rangle \}$
  \item $v(a) = \partial_a$ usw., $v_{0}(P) = $
  \end{itemize}
  \pause Welche Formeln gelten an der jeweiligen Welt?
  \begin{columns}
    \begin{column}[t]{.5\linewidth}

      \begin{itemize}[<+->]
      \item $\Box Pa$ an $w_1$
      \item $\exists x \Box Px$, $w_0$
      \item $\Box \exists x Px$, $w_0$
      \end{itemize}

    \end{column}

    \begin{column}[t]{.5\linewidth}
      \begin{itemize}[<+->]
      \item $\forall x \Diamond Px$, $w_0$
      \item $\Diamond \forall x Px$, $w_0$
      \end{itemize}
    \end{column}
  \end{columns}

\end{frame}

\begin{frame}{Tableaux-Regeln}

  Alle Regeln der klassischen Prädikatenlogik, relativ zu einer Welt,
  und die Regeln der modalen Aussagenlogik.

  \pause

  \begin{columns}
    \begin{column}[t]{.25 \linewidth}
      \begin{center}
        $\forall x A, i$ \\
        \prule\\
        $A_x(a), i$
      \end{center}
      ($a$ kam schon vor.)
    \end{column}
    
    \begin{column}[t]{.25 \linewidth}
      \begin{center}
        $\exists x A, i$ \\
        \prule\\
        $A_x(c), i$
      \end{center}
      ($c$ ist eine neue Konstante.)
    \end{column}
    
    \begin{column}[t]{.25 \linewidth}
      \begin{center}
        $\neg \exists x A, i$\\
        \prule\\
        $\forall x \neg A, i$\\
      \end{center}
    \end{column}
    
    \begin{column}[t]{.25 \linewidth}
      \begin{center}
        $\neg \forall x A, i$ \\
        \prule\\
        $\exists x \neg A, i$
      \end{center}
    \end{column}
  \end{columns}

  \pause
  
  \begin{columns}
    \begin{column}[t]{.5 \linewidth}
      \begin{center}
        $\Box A, i$ \\
        $irj$ \\
        \prule \\
        $A, j$
      \end{center}
    \end{column}
    \begin{column}[t]{.5 \linewidth}
      \begin{center}
        $\Diamond A, i$\\
        \prule \\
        $irj$ \\
        $A,j$ \\
        ($j$ ist neu.)
      \end{center}

    \end{column}
  \end{columns}

  
\end{frame}



\begin{frame}{Tableaux: Beispiele}

$K$ ist eine Logik ohne Beschränkung der Relation $R$ der Interpretationen.
  \begin{enumerate}[<+->]
  \item   $\vdash_K \forall x \Box Px \to \Box \forall x Px$? 
  \item $\vdash_K \Box \forall x Px \to \forall x \Box Px$?
  \item $\vdash_K \exists x \Box Px \to \Box \exists x Px$?
  \item $\vdash_K \Box \exists x Px \to \exists x \Box Px$?
  \end{enumerate}


\end{frame}


\begin{frame}
  \Huge Philosophische Probleme und Fragen
\end{frame}

\begin{frame}{Barcan-Formel}

  \begin{itemize}[<+->]
  \item Constant Domain Prädikatenlogik kommt mit einem heftigen Commitment:
    \begin{center}
       Barcan Formula (BF): $\forall x \Box Px \to \Box \forall x Px$
    \end{center}
  \item BF ist Theorem von K (und jeder stärkeren Logik).
  \item Wie lässt sich das interpretieren? Nehme an, das Antezedens von BF gilt an $w_0$: Alle Dinge an $w_0$ haben die Eigenschaft $P$ notwendigerweise. Dann gilt nach BF auch das Konsequenz: An allen Welten gilt $\forall x Px$. 
  \end{itemize}
  
\end{frame}

\end{document}
