\documentclass[12pt]{beamer}
\usepackage[utf8]{inputenc}
\usecolortheme{dove}
% \setbeamercovered{transparent}


\newcommand{\prule}{\vspace{-8pt}\rule{50pt}{0.5pt}}




\title[Philosophische modale Prädikatenlogik]{Modale Prädikatenlogik}
\subtitle{Eine sehr kurze Einführung}
\author{Conrad Friedrich}
\institute[Uni Köln]{Universität zu Köln}
\date{\today}



\begin{document}

\begin{frame}
\titlepage
\end{frame}

\section{Prädikatenlogik}

\begin{frame}
  \Huge Prädikatenlogik \\
\pause
\large ``Baby Logic" Version
\end{frame}

\begin{frame}{Formalisierung}
  \begin{columns}

    \begin{column}[t]{5cm}
      \begin{enumerate}
      \item<1-> Alle Menschen sind sterblich \\
      \item<1-> Sokrates ist ein Mann\\
      \item<2-> Also: Sokrates ist sterblich
      \end{enumerate}
    \end{column}

    \begin{column}[t]{5cm}
      \begin{enumerate}
      \item<3-> $P$
      \item<3-> $Q$
      \item<3-> Also: $R$
      \end{enumerate}
    \end{column}

  \end{columns}

  \vspace{10 mm}
  \begin{itemize}
  \item<4-> Mehr Struktur, als wir mit der Aussagenlogik abbilden
    können
  \end{itemize}
\end{frame}

\begin{frame}{Stattdessen: Formalisierung II}
  
  \begin{columns}

    \begin{column}[t]{5cm}
      \begin{enumerate}
      \item<1-> Alle Menschen sind sterblich.
      \item<1-> Sokrates ist ein Mensch.
      \item<1-> Also: Sokrates ist sterblich.
      \end{enumerate}
    \end{column}

    \begin{column}[t]{5cm}
      \begin{enumerate}
      \item<2-> $\forall x (Mx \to Sx)$.
      \item<3-> $Ms$.
      \item<4-> Also: $Ss$.
      \end{enumerate}
    \end{column}

  \end{columns}

\end{frame}

\begin{frame}{Formalisierung III}
  \begin{itemize}
  \item<1-> Es gibt genau einen Gott.
  \item<2-> gdw. Es gibt ein Ding, das Gott ist, und alle anderen
    Dinge sind, falls sie Gott sind, identisch mit diesem Ding.
  \item<3-> gdw. $\exists x(Gx \land \forall y(Gx \to x=y))$.
  \end{itemize}
\end{frame}

\begin{frame}{Vokabular}
  \begin{itemize}
  \item<1-> Variablen
    \begin{itemize}
    \item $\forall \mathbf{x} (M\mathbf{x} \to S\mathbf{x})$
    \end{itemize}
  \item<2-> Konstanten
    \begin{itemize}
    \item $P\mathbf{c}$
    \end{itemize}
  \item<3-> Prädikatensymbole (n-stellig)
    \begin{itemize}
    \item $\forall x (\mathbf{M}x \to \mathbf{S}x)$, $\mathbf{G}xy$
    \end{itemize}
  \item<4-> Konnektive wie in der Aussagenlogik
    \begin{itemize}
    \item $\neg, \to, \land, \lor, \leftrightarrow$
    \end{itemize}
  \item<5-> Quantoren
    \begin{itemize}
    \item $\forall, \exists$
    \end{itemize}
  \end{itemize}
\end{frame}


\begin{frame}{Grammatik}
  \begin{enumerate}
  \item<1-> Wenn $t_1,...,t_n$ Variablen oder Konstanten sind und $P$
    ein $n$-stelliges Prädikat ist, dann ist $Pt_1,...,t_n$ eine
    (atomare, wohlgeformte) Formel.
    \begin{itemize}
    \item<2-> $Fx$, $Rab$
    \end{itemize}
  \item<3-> Wenn $A$ und $B$ Formeln sind, dann sind auch $\neg A$,
    $A\to B$, $A \land B$, $A \lor B$, $A \leftrightarrow B$ Formeln.
  \item<4-> Wenn A eine Formel ist und $x$ eine Variable, dann sind
    $\forall xA$ und $\exists x A$ Formeln.
  \end{enumerate}
\end{frame}

\begin{frame}{Grammatik II: Beispiele}
  Seien $a,b,c$ Konstanten und $P,Q$ Prädikatensymbole. Was wird
  intuitiv mit diesen Formeln ausgedrückt?

  \begin{itemize}
  \item<2-> $Pa$, $Qab$
  \item<3-> $\neg Pa$
  \item<4-> $Qaa$
  \item<5-> $Qab \leftrightarrow Qba$
  \item<6-> $(Qab \land Qbc) \to Qac$
  \end{itemize}

\end{frame}

\begin{frame}{Grammatik III: Beispiele}
  Seien $x, y, z$ Variablen, $a,b,c$ Konstanten und $P,Q$
  Prädikatensymbole. Was wird intuitiv mit diesen Formeln ausgedrückt?

  \begin{itemize}
  \item<2-> $Px$, $Qxy$
  \item<3-> $\exists x Px$
  \item<4-> $\forall x (Px \to Qax)$
  \item<5-> $\exists y Qxy$
  \item<6-> $\forall x \exists y Qxy$
  \end{itemize}

  \begin{itemize}
  \item<7-> Variablen, die in einer Formel im Skopus eines Quantors
    stehen, sind \emph{gebunden}, sonst \emph{frei}.
  \item<8-> Formeln, in denen keine freien Variablen vorkommen, heißen
    \emph{Sätze}.
  \end{itemize}
\end{frame}

\begin{frame}{Semantik}
  \begin{itemize}[<+->]
  \item Wann würden wir $Qab$ intuitiv als wahr bezeichnen?

  \item Wenn die Dinge, für die $a,b$ stehen, tatsächlich die
    Eigenschaft haben, die mit $Q$ bezeichnet wird.
  \item $Px$?

  \item Die Variable $x$ hat die Eigenschaft $P$?
  \item Nonsense. Wir können nur \emph{Sätzen} Wahrheitswerte zuordnen
    (zumindest ohne Weiteres.)
  \end{itemize}
\end{frame}

\begin{frame}{Semantik II}
  \begin{itemize}[<+->]
  \item Wann würden wir intuitiv $\exists x Px$ als wahr bezeichnen?
  \item Wenn \emph{irgendein} Ding $P$ erfüllt.
  \item $\forall x Fx$?
  \item Wenn \emph{alle} Dinge $F$ erfüllen.
  \end{itemize}
\end{frame}

\begin{frame}{Semantik III - Interpretation}
  Eine Interpretation $I$ besteht aus einem Tupel
  $\langle D,v\rangle$.
  \begin{itemize}[<+->]
  \item $D$ ist der (nicht-leere) Gegenstandsbereich, über die
    quantifiziert wird.
  
  \item $v$ ist eine Funktion, so dass
    \begin{itemize}
    \item Wenn $c$ eine Konstante ist, dann ist $v(c) \in D$.
    \item Wenn $P$ ein $n$-stelliges Prädikatensymbol ist, dann ist
      $v(P) \subseteq D^n$.
    \end{itemize}
  \item (Tafel)
  \end{itemize}

\end{frame}

\begin{frame}{Semantik IV: Wahrheit in einer Interpretation}
  \begin{itemize}[<+->]
  \item $v(Pc_1,...c_n) = 1$ gdw.
    $\langle v(c_1),...,v(c_n)\rangle \in v(P)$, sonst 0.
    \begin{itemize}
    \item $Fa$ ist wahr gdw.\\ $v(a)$ (das Objekt von $a$) in $v(F)$
      (der Extension von $F$) enthalten ist
    \end{itemize}
  \item $v(\neg A)$, $v(A \land B)$ usw. genau wie in der
    Aussagenlogik.
  \end{itemize}

  \begin{itemize}[<+->]
    
\item $v(\forall xA) = 1$ gdw. \textbf{jedes} Objekt des Gegenstandsbereiches $A$ erfüllt, sonst 0.
\item $v(\exists xA) = 1$ gdw. \textbf{mindestens} ein Objekt des Gegenstandsbereiches $A$ erfüllt, sonst 0.
  \begin{itemize}
  \item 
  \end{itemize}
  % \item $v(\forall xA) = 1$ gdw. \textbf{für alle} $\partial \in D$
  %   gilt: $v(A_x(c_\partial)) = 1$, sonst 0.
  % \item $v(\exists xA) = 1$ gdw. \textbf{für mindestens ein}
  %   $\partial \in D$ gilt: $v(A_x(c_\partial)) = 1$, sonst 0.

  \end{itemize}

\end{frame}

% \begin{frame}{Semantik V: Wahrheit in einer Interpretation II}
  % \begin{block}{Notation} Sei $\partial \in D$ und $c_\partial$ eine
  %   Konstante, sodass $v(c_\partial) = \partial$.  Dann ist
  %   $A_x(c_\partial)$ die Formel, die wir erhalten, wenn wir alle
  %   Vorkommnisse von $x$ in $A$ durch $c_\partial$ ersetzen.
  %   \\
  %   \pause
  %   \begin{itemize}
  %   \item Beispiel: $A$ sei $Px \land Qy$. $A_x(c_\partial)$ ist dann
  %     $Pc_\partial \land Qy$.
  %   \end{itemize}
  % \end{block}
  % \pause

% \end{frame}

\begin{frame}{Semantik VI: Interpretation Beispiel}

  Konstanten: $a, b, c$. Prädikatensymbole: $P, Q$. \\
  Sei $I$ gegeben durch: $D = \{\partial_a, \partial_b, \partial_c\}$,\\
  $v(a) = \partial_a$ usw., \\
  $v(P) = \{\partial_a, \partial_b\}$, $v(Q) = \{\langle \partial_a, \partial_a \rangle\, \langle \partial_c, \partial_b \rangle \}$.\\
  Welche der folgenden Formeln gilt in $I$?

  \pause

  \begin{itemize}[<+->]
  \item $Pa \lor Qac$.
  \item $\exists x (Qxx \land Px)$.
  \item $\forall x (Px \to \exists y Qxy)$.
  \item $\forall x (Px \lor Qxy).$
  \end{itemize}

\end{frame}

\begin{frame}{Tableaux-Regeln}
  \begin{columns}

    \begin{column}[t]{.25 \linewidth}
      \begin{center}
        $\forall x A$ \\
        \prule\\
        $A_x(a)$
      \end{center}
      $a$ ist eine Konstante, die schon vorkam.
    \end{column}
    \pause
    \begin{column}[t]{.25 \linewidth}
      \begin{center}
        $\exists x A$ \\
        \prule\\
        $A_x(c)$
      \end{center}
      $c$ ist eine neue Konstante.
    \end{column}
    \pause
    \begin{column}[t]{.25 \linewidth}
      \begin{center}
        $\neg \exists x A$\\
        \prule\\
        $\forall x \neg A$\\
      \end{center}
    \end{column}
    \pause
    \begin{column}[t]{.25 \linewidth}
      \begin{center}
        $\neg \forall x A$ \\
        \prule\\
        $\exists x \neg A$
      \end{center}
    \end{column}
  \end{columns}

  \vspace{1em}
  \begin{columns}

    \pause

    \begin{column}[t]{.5 \linewidth}
      \begin{itemize}[<+->]
      \item $Pc \vdash \exists x Px$?
      \end{itemize}
    \end{column}

    \begin{column}[t]{.5 \linewidth}
      \begin{itemize}[<+->]
      \item $\forall x \neg Px \vdash \neg \exists x Px$?
      \item $\exists x \neg Px \vdash \neg \forall x Px$?
      \end{itemize}

      \end{column}
  \end{columns}

\end{frame}




\end{document}
