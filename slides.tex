
\documentclass[12pt]{beamer}
\usepackage[utf8]{inputenc}
\usecolortheme{default}
\usepackage{graphicx}
% \setbeamercovered{transparent}


\newcommand{\prule}{\vspace{-8pt}\rule{50pt}{0.5pt}}




\title[Modale Prädikatenlogik]{Philosophische Modale Prädikatenlogik}
\subtitle{Eine sehr kurze Einführung}
\author{Conrad Friedrich}
\institute[Uni Köln]{Universität zu Köln}
\date{\today}



\begin{document}

\begin{frame}
\titlepage
\end{frame}

\section{Prädikatenlogik}

\begin{frame}
  \Huge Prädikatenlogik \\
  \pause \large ``Baby Logic" Version\\
  \pause \large Notation stark an Priest (2008) angelehnt.
\end{frame}

\begin{frame}{Formalisierung}
  \begin{columns}

    \begin{column}[t]{5cm}
      \begin{enumerate}[<+->]
      \item Alle Menschen sind sterblich. \\
      \item Sokrates ist ein Mensch.\\
      \item Also: Sokrates ist sterblich.
      \end{enumerate}
    \end{column}

    \begin{column}[t]{5cm}
      \begin{enumerate}[<+->]
      \item $P$
      \item $Q$
      \item Also: $R$

      \end{enumerate}

      \pause

      \begin{center}
        \includegraphics[height=2cm]{scepticdog.png}
      \end{center}
    \end{column}

  \end{columns}

  \pause

  \vspace{10 mm}
  \begin{itemize}
  \item<4-> Mehr Struktur, als wir mit der Aussagenlogik abbilden
    können
  \end{itemize}
\end{frame}

\begin{frame}{Formalisierung}
  
  \begin{columns}

    \begin{column}[t]{5cm}
      \begin{enumerate}[<+->]
      \item Alle Menschen sind sterblich.
      \item Sokrates ist ein Mensch.
      \item Also: Sokrates ist sterblich.
      \end{enumerate}
    \end{column}

    \begin{column}[t]{5cm}
      \begin{enumerate}[<+->]
      \item $\forall x (Mx \to Sx)$
      \item $Ms$
      \item Also: $Ss$

      \end{enumerate}
      \pause
      \begin{center}
        \includegraphics[height=2cm]{cleverdog.png}

      \end{center}

    \end{column}

  \end{columns}

\end{frame}

\begin{frame}{Formalisierung}
  \begin{itemize}
  \item<1-> Es gibt genau einen Gott.
  \item<2-> gdw. Es gibt ein Ding, das Gott ist, und alle anderen
    Dinge sind, falls sie Gott sind, identisch mit diesem Ding.
  \item<3-> gdw. $\exists x(Gx \land \forall y(Gx \to x=y))$.
  \end{itemize}
\end{frame}

\begin{frame}{Vokabular}
  \begin{itemize}
  \item<1-> Variablen
    \begin{itemize}
    \item $\forall \mathbf{x} (M\mathbf{x} \to S\mathbf{x})$
    \end{itemize}
  \item<2-> Konstanten
    \begin{itemize}
    \item $P\mathbf{c}$
    \end{itemize}
  \item<3-> Prädikatensymbole (n-stellig)
    \begin{itemize}
    \item $\forall x (\mathbf{M}x \to \mathbf{S}x)$, $\mathbf{G}xy$
    \end{itemize}
  \item<4-> Konnektive wie in der Aussagenlogik
    \begin{itemize}
    \item $\neg, \to, \land, \lor, \leftrightarrow$
    \end{itemize}
  \item<5-> Quantorsymbole
    \begin{itemize}
    \item $\forall, \exists$
    \end{itemize}
  \item<6-> (Funktionssymbole, Hilfszeichen...)
  \end{itemize}
\end{frame}


\begin{frame}{Grammatik}
  \begin{enumerate}
  \item<1-> Wenn $t_1,...,t_n$ Variablen oder Konstanten sind und $P$
    ein $n$-stelliges Prädikat ist, dann ist $Pt_1,...,t_n$ eine
    (atomare, wohlgeformte) Formel.
    \begin{itemize}
    \item<2-> $Fx$, $Rab$
    \end{itemize}
  \item<3-> Wenn $A$ und $B$ Formeln sind, dann sind auch $\neg A$,
    $A\to B$, $A \land B$, $A \lor B$, $A \leftrightarrow B$ Formeln.
  \item<4-> Wenn A eine Formel ist und $x$ eine Variable, dann sind
    $\forall xA$ und $\exists x A$ Formeln.
  \end{enumerate}
\end{frame}

\begin{frame}{Grammatik: Beispiele}
  Seien $a,b,c$ Konstanten und $P,Q$ Prädikatensymbole. Wie werden
  diese Formeln gelesen?

  \begin{itemize}
  \item<2-> $Pa$, $Qab$
  \item<3-> $\neg Pa$
  \item<4-> $Qaa$
  \item<5-> $Qab \leftrightarrow Qba$
  \item<6-> $(Qab \land Qbc) \to Qac$
  \end{itemize}

\end{frame}

\begin{frame}{Grammatik: Beispiele}
  Seien $x, y, z$ Variablen, $a,b,c$ Konstanten und $P,Q$
  Prädikatensymbole. Wie werden diese Formeln gelesen?

  \begin{itemize}
  \item<2-> $Px$, $Qxy$
  \item<3-> $\exists x Px$
  \item<4-> $\forall x (Px \to Qax)$
  \item<5-> $\exists y Qxy$
  \item<6-> $\forall x \exists y Qxy$
  \end{itemize}

  \begin{itemize}
  \item<7-> Variablen, die in einer Formel im Skopus eines zugehörigen
    Quantors stehen, sind \emph{gebunden}, sonst \emph{frei}.
  \item<8-> Formeln, in denen keine freien Variablen vorkommen, heißen
    \emph{Sätze}.
  \end{itemize}
\end{frame}

\begin{frame}{Tableaux-Regeln}

\begin{block}{Notation}
  $A_x(c)$ ist die Formel, die wir erhalten, wenn wir alle freien
  Vorkommnisse von $x$ in $A$ durch $c$ ersetzen.
\end{block}
\pause
\begin{columns}

  \begin{column}[t]{.25 \linewidth}
    \begin{center}
      $\forall x A$ \\
      \prule\\
      $A_x(a)$
    \end{center}
    $a$ ist eine Konstante, die schon vorkam.
  \end{column}
  \pause
  \begin{column}[t]{.25 \linewidth}
    \begin{center}
      $\exists x A$ \\
      \prule\\
      $A_x(c)$
    \end{center}
    $c$ ist eine neue Konstante.
  \end{column}
  \pause
  \begin{column}[t]{.25 \linewidth}
    \begin{center}
      $\neg \exists x A$\\
      \prule\\
      $\forall x \neg A$\\
    \end{center}
  \end{column}
  \pause
  \begin{column}[t]{.25 \linewidth}
    \begin{center}
      $\neg \forall x A$ \\
      \prule\\
      $\exists x \neg A$
    \end{center}
  \end{column}
\end{columns}

\vspace{1em}
\begin{columns}

  \pause

  \begin{column}[t]{.5 \linewidth}
    \begin{itemize}[<+->]
    \item $Pc \vdash \exists x Px$?
    \end{itemize}
  \end{column}

  \begin{column}[t]{.5 \linewidth}
    \begin{itemize}[<+->]
    \item $\forall x \neg Px \vdash \neg \exists x Px$?
    \item $\exists x \neg Px \vdash \neg \forall x Px$?
    \end{itemize}

  \end{column}
\end{columns}

\end{frame}



\begin{frame}{Semantik}
  \begin{itemize}[<+->]
  \item Wann würden wir $Qab$ intuitiv als wahr bezeichnen?

  \item Wenn die Dinge, für die $a,b$ stehen, tatsächlich die
    Eigenschaft haben, die mit $Q$ bezeichnet wird.
  \item $Px$?

  \item Die Variable $x$ hat die Eigenschaft $P$?
  \item Nonsense. Wir können nur \emph{Sätzen} Wahrheitswerte zuordnen
    (zumindest ohne Weiteres.)
  \end{itemize}
\end{frame}

\begin{frame}{Semantik}
  \begin{itemize}[<+->]
  \item Wann würden wir intuitiv $\exists x Px$ als wahr bezeichnen?
  \item Wenn \emph{irgendein} Ding $P$ erfüllt.
  \item $\forall x Fx$?
  \item Wenn \emph{alle} Dinge $F$ erfüllen.
  \end{itemize}
\end{frame}

\begin{frame}{Semantik: Interpretation}
  Eine Interpretation $I$ besteht aus einem Tupel
  $\langle D,v\rangle$.
  \begin{itemize}[<+->]
  \item $D$ ist der (nicht-leere) Gegenstandsbereich, über die
    quantifiziert wird.
  
  \item $v$ ist eine Funktion, so dass
    \begin{itemize}
    \item Wenn $c$ eine Konstante ist, dann ist $v(c) \in D$.
    \item Wenn $P$ ein $n$-stelliges Prädikatensymbol ist, dann ist
      $v(P) \subseteq D^n$.
    \end{itemize}
  \item (Tafel)
  \end{itemize}

\end{frame}

\begin{frame}{Semantik: Wahrheit in einer Interpretation}
  \begin{itemize}[<+->]
  \item $v(Pc_1,...c_n) = 1$ gdw.
    $\langle v(c_1),...,v(c_n)\rangle \in v(P)$, sonst 0.
    \begin{itemize}
    \item $Fa$ ist wahr gdw.\\ $v(a)$ (das Objekt von $a$) in $v(F)$
      (der Extension von $F$) enthalten ist
    \end{itemize}
  \item Restliche Konnektive genau wie in der Aussagenlogik.
    \begin{itemize}[<+->]
    \item $v(\neg A) = 1$ gdw $v(A) = 0$, sonst 0.
    \item $v(A \land B) = 1$ gdw $v(A) = 1$ und $v(B) = 1$, sonst 0.
    \item usw.
    \end{itemize}
  \end{itemize}

  \begin{itemize}[<+->]
    
  \item $v(\forall xA) = 1$ gdw. \textbf{jedes} Objekt des
    Gegenstandsbereiches $A$ erfüllt, sonst 0.
  \item $v(\exists xA) = 1$ gdw. \textbf{mindestens ein} Objekt des
    Gegenstandsbereiches $A$ erfüllt, sonst 0.
    \begin{itemize}
    \item Eine Formel $A$ gilt in $I$ gdw. $v(I) = 1$.
    \item (Formal unterbestimmt. Was heißt `erfüllen'?)
    \end{itemize}
    % \item $v(\forall xA) = 1$ gdw. \textbf{für alle}
    %   $\partial \in D$
    %   gilt: $v(A_x(c_\partial)) = 1$, sonst 0.
    % \item $v(\exists xA) = 1$ gdw. \textbf{für mindestens ein}
    %   $\partial \in D$ gilt: $v(A_x(c_\partial)) = 1$, sonst 0.

  \end{itemize}

\end{frame}

% \begin{frame}{Semantik V: Wahrheit in einer Interpretation II}
%   \begin{block}{Notation} Sei $\partial \in D$ und $c_\partial$ eine
%     Konstante, sodass $v(c_\partial) = \partial$.  Dann ist
%     $A_x(c_\partial)$ die Formel, die wir erhalten, wenn wir alle
%     Vorkommnisse von $x$ in $A$ durch $c_\partial$ ersetzen.
%     \\
%     \pause
%     \begin{itemize}
%     \item Beispiel: $A$ sei $Px \land Qy$. $A_x(c_\partial)$ ist
%       dann $Pc_\partial \land Qy$.
%     \end{itemize}
%   \end{block}
%   \pause

% \end{frame}

\begin{frame}{Semantik: Interpretation Beispiel}

  Konstanten: $a, b, c$. Prädikatensymbole: $P, Q$. \\
  Sei $I$ gegeben durch: $D = \{\partial_a, \partial_b, \partial_c\}$,\\
  $v(a) = \partial_a$ usw., \\
  $v(P) = \{\partial_a, \partial_b\}$, $v(Q) = \{\langle \partial_a, \partial_a \rangle\, \langle \partial_c, \partial_b \rangle \}$.\\
  Welche der folgenden Formeln gilt in $I$?

  \pause

  \begin{itemize}[<+->]
  \item $Pa \lor Qac$.
  \item $\exists x (Qxx \land Px)$.
  \item $\forall x (Px \to \exists y Qxy)$.
  \item $\forall x (Px \lor Qxy).$
  \end{itemize}

\end{frame}


\begin{frame}
  \Huge Modale Prädikatenlogik\\
  \large Constant Domain
\end{frame}

\begin{frame}{Vokabular und Grammatik}
  \begin{block}{Vokabular}
    Zum Vokabular werden $\Box$ und $\Diamond$ hinzugefügt.
  \end{block}

  \pause
  \begin{block}{Grammatik}
    Wir erweitern die Grammatik der Prädikatenlogik, so dass: \pause
    \begin{itemize}[<+->]
    \item Wenn $A$ eine Formel ist, dann ist auch $\Box A$ eine
      Formel.
    \item Wenn $A$ eine Formel ist, dann ist auch $\Diamond A$ eine
      Formel.
      \begin{itemize}
      \item $\forall x \Box (Px \land Qx) \to \Box \forall x Px$
      \item
        $\Box \Diamond \exists x Px \to \Box \exists x \Diamond (Px
        \lor Qx)$
      \end{itemize}
    \item Restliche Grammatik wie in der klassischen Prädikatenlogik.
    \end{itemize}
  \end{block}

\end{frame}

\begin{frame}{Tableaux: Regeln}

  Alle Regeln der klassischen Prädikatenlogik, relativ zu einer Welt,
  und die Regeln der modalen Aussagenlogik.

  \pause

  \begin{columns}
    \begin{column}[t]{.25 \linewidth}
      \begin{center}
        $\forall x A, i$ \\
        \prule\\
        $A_x(a), i$
      \end{center}
      ($a$ kam schon vor.)
    \end{column}
    
    \begin{column}[t]{.25 \linewidth}
      \begin{center}
        $\exists x A, i$ \\
        \prule\\
        $A_x(c), i$
      \end{center}
      ($c$ ist eine neue Konstante.)
    \end{column}
    
    \begin{column}[t]{.25 \linewidth}
      \begin{center}
        $\neg \exists x A, i$\\
        \prule\\
        $\forall x \neg A, i$\\
      \end{center}
    \end{column}
    
    \begin{column}[t]{.25 \linewidth}
      \begin{center}
        $\neg \forall x A, i$ \\
        \prule\\
        $\exists x \neg A, i$
      \end{center}
    \end{column}
  \end{columns}

  \pause
  
  \begin{columns}
    \begin{column}[t]{.5 \linewidth}
      \begin{center}
        $\Box A, i$ \\
        $irj$ \\
        \prule \\
        $A, j$
      \end{center}
    \end{column}
    \begin{column}[t]{.5 \linewidth}
      \begin{center}
        $\Diamond A, i$\\
        \prule \\
        $irj$ \\
        $A,j$ \\
        ($j$ ist neu.)
      \end{center}

    \end{column}
  \end{columns}

  
\end{frame}



\begin{frame}{Tableaux: Beispiele}

  $K$ ist eine Constant Domain Logik ohne Beschränkung der Relation
  $R$ der Interpretationen.
  \begin{enumerate}[<+->]
  \item $\vdash_K \forall x \Box Px \to \Box \forall x Px$?
  \item $\vdash_K \Box \forall x Px \to \forall x \Box Px$?
  \item $\vdash_K \exists x \Diamond Px \to \Diamond \exists x Px$?
  \item $\vdash_K \Diamond \exists x Px \to \exists x \Diamond Px$?
  \end{enumerate}


\end{frame}

\begin{frame}{Semantik: Intuitiv}
  Wir wollen die Logik so erweitern, dass: \pause
  \begin{itemize}[<+->]
  \item Ein Gegenstand an einer SituationWelt eine Eigenschaft haben
    kann, aber an einer anderen Welt nicht.
  \item $v(Pa) = 1$ an $w_0$, aber $v(Pa) = 0$ an $w_1$.
  \end{itemize}
\end{frame}

\begin{frame}{Semantik: Interpretation}
  Die Interpretation $I$ wird erweitert, so dass sie aus einem 4-tupel
  $\langle D, W, R, v \rangle$ besteht, wobei \pause
  \begin{itemize}[<+->]
  \item $D$ wie in der klassischen Prädikatenlogik der nicht-leere
    Gegenstandsbereich ist,
  \item $W$ wie in der modalen Aussagenlogik eine Menge möglicher
    Welten ist,
  \item $R$ wie in der modalen Aussagenlogik eine Relation auf $W$ ist
    (d.h. $R \subseteq W \times W$),
  \item $v$ eine Funktion ist, so dass
    \begin{itemize}
    \item Wenn $c$ eine Konstante ist, dann ist $v(c) \in D$.
    \item Wenn $P$ ein $n$-stelliges Prädikatensymbol ist und
      $w \in W$, dann ist $v_w(P) \subseteq D^n$.
    \end{itemize}
  \end{itemize}
\end{frame}

\begin{frame}{Semantik}
  Eine Formel ist nun wahr oder falsch in einer Interpretation
  \emph{an einer Welt}.  Also
  \begin{itemize}[<+->]
  \item $v_\mathbf{w}(Pc_1,...c_n) = 1$ gdw.
    $\langle v(c_1),...,v(c_n)\rangle \in v_\mathbf{w}(P)$, sonst 0.
  \item Ganz analog mit nicht atomaren Formeln:
    \begin{itemize}
    \item $v_w(\neg A) = 1$ gdw. $v_w(A) = 0$.
    \item $v_w(A \land B) = 1$ gdw. $v_w(A) = 1$ und $v_w(B) = 1$,
      sonst 0.
    \item usw.
    \end{itemize}
  \item $v_w(\Box A) = 1$ gdw. \textbf{für alle} $w' \in W$ mit $wRw'$
    gilt: $v_{w'}(A) = 1$.
  \item $v_w(\Diamond A) = 1$ gdw. \textbf{für mindestens ein}
    $w' \in W$ mit $wRw'$ gilt: $v_{w'}(A) = 1$.
  \end{itemize}
\end{frame}

\begin{frame}{Semantik: Beispiel}
  Sei eine Interpretation $I$ gegeben durch: \pause
  \begin{itemize}[<+->]
  \item $D = \{\partial_a, \partial_b\}$
  \item $W = \{w_0, w_1\}$
  \item $R = \{\langle w_0, w_0\rangle ,\langle w_0, w_1 \rangle \}$
  \item $v(a) = \partial_a$, $v(b) = \partial_b$,
  \item $v_{0}(P) = \{\partial_a\}$, $v_1(P) = \{\partial_b\}$
  \end{itemize}
  \pause Welche Formeln gelten an der jeweiligen Welt?
  \begin{columns}
    \begin{column}[t]{.5\linewidth}

      \begin{itemize}[<+->]
      \item $\Box Pa$ an $w_0, w_1$
      \item $\exists x \Box Px$, $w_0$
      \item $\Box \exists x Px$, $w_0$
      \end{itemize}

    \end{column}

    \begin{column}[t]{.5\linewidth}
      \begin{itemize}[<+->]
      \item $\forall x \Diamond Px$, $w_0$
      \item $\Diamond \forall x Px$, $w_0$
      \end{itemize}
    \end{column}
  \end{columns}

\end{frame}


\begin{frame}
  \Huge Philosophische Probleme: Barcan Formula
\end{frame}

\begin{frame}{Barcan Formula}

  \begin{itemize}[<+->]
  \item Constant Domain Prädikatenlogik kommt mit einem heftigen
    Commitment:
    \begin{center}
      Barcan Formula (BF): $\forall x \Box A \to \Box \forall x A$
    \end{center}
  \item BF ist Theorem von K (und jeder stärkeren Logik).
  \item Wie lässt sich das interpretieren? Nehme an, das Antezedens
    von BF gilt an $w_0$: Alle Dinge an $w_0$ haben die Eigenschaft
    $P$ an allen (zugänglichen) Welten. Dann gilt nach BF auch das
    Konsequenz: An allen Welten gilt $\forall x Px$. D.h. an jeder
    Welt gilt $Px$ für alle \emph{dort} existierenden Dinge.
  \item Ist das plausibel? Was ist, wenn an einer Welt $w_1$ andere
    Dinge existieren als an Welt $w_0$? Dann dürfte
    $\forall x \Box Px$ an $w_0$ nichts über diese Dinge
    implizieren. \end{itemize}
  
\end{frame}

\begin{frame}{Barcan Formula}

  \begin{itemize}
  \item Äquivalent zur Barcan Formula:
    \begin{center}
      BF*: $\Diamond \exists x A \to \exists x \Diamond A$
    \end{center}
  \item BF* auch Theorem von K.
  \item Intuitiv interpretieren: Wenn es an dieser Welt möglich ist,
    dass es etwas gibt, das die Eigenschaft $P$ hat, dann gibt es
    etwas an dieser Welt, von dem es möglich ist, dass es die
    Eigenschaft $P$ hat.
  \item D.h. Wenn an irgendeiner (zugänglichen) Welt irgendetwas eine
    völlig obskure Eigenschaft, dann gibt es an dieser Welt etwas, das
    möglicherweise diese Eigenschaft hat.
  
  \end{itemize}
\end{frame}

\begin{frame}{Barcan Formula: Beispiel}
  \begin{itemize}[<+->]
  \item Beispiel: Wenn es an irgendeiner Welt ein Perpetuum Mobile
    gibt, dann gibt es an dieser Welt ein Gerät, das möglicherweise
    ein Perpetuum Mobile ist. D.h. dieses Gerät aus unserer Welt wäre
    an einer anderen Welt ein Perpetuum Mobile. Aber würden wir wir
    dann noch von demselben Gerät sprechen?
  \item Man würde vielleicht gerne so etwas sagen wie: Für alle Dinge
    auf \emph{dieser} Welt ist es unmöglich, ein Perpetuum Mobile zu
    sein. Formalisiert: $\forall x \Box \neg Px$.
  \item An anderen Welten soll es ruhig Dinge geben können, die ein
    Perpettum Mobile sind.
  \end{itemize}
\end{frame}

\begin{frame}{Reverse Barcan Formula}
  \begin{itemize}[<+->]
  \item Aber dann beißt uns die Reverse Barcan Formula:
    \begin{center}
      Reverse Barcan Formula (RBF):
      $\forall x \Box A \to \Box \forall x A$.
    \end{center}
  \item RBF ist \emph{auch} ein Theorem von K.
  \item D.h. sobald wir behaupten dass alle Dinge in dieser Welt
    notwendigerweise kein Perpetuum Mobile sind, folgt, dass es an
    \emph{keiner} möglichen Welt ein Perpetuum Mobile gibt. Aber das
    wollten wir ja gerade nicht ausschließen!
  \item Es scheint: Unser intuitives Verständnis von Notwendigkeit
    wird hier nicht adäquat mit dem formalen Modell eingefangen.
   
  \end{itemize}
  \pause
  \begin{center}
    \includegraphics[height=3cm]{cryingdog.png}
  \end{center}
\end{frame}

\begin{frame}{Barcan Formula: Responses}
  Welche Möglichkeiten gibt es, Constant Domain Modal Predicate Logic
  gegen den BF-Einwand zu verteidigen?
  \begin{enumerate}[<+->]
  \item[1.] Implizit habe ich von allen möglichen Welten gesprochen,
    aber eigentlich geht es ja nur um die zugänglichen (R-Relation).
    \begin{itemize}[<+->]
    \item So gesehen kann $\forall x \Box \neg Px$ an $@$ gelten, und
      es trotzdem Welten mit $\exists x Px$ geben. Die sind aber von
      $@$ aus unerreichbar, und daher für modale Aussagen in der
      aktualen Welt vollkommen irrelevant! Das Problem tritt außerdem
      immer noch für allen erreichbaren Welten auf.
    \end{itemize}
  \end{enumerate}
\end{frame}
\begin{frame}{Barcan Formula: Responses}
  Um Constant Domain zu verteidigen könnte auch sagen:
  \begin{enumerate}[<+->]
  \item[2.] Unser intuitives Verständnis der Quantoren ist
    falsch. Bisher $\exists x A$ als ``es existiert ein x, sodass A''
    gelesen. Stattdessen: Quantoren quantifizieren nicht über alle
    existenten Dinge, sondern über alle möglichen Dinge ($\implies$
    possibilistische Quantifizierung). BF klingt dann intuitiv
    plausibel, weil die Quantoren so gelesen werden, dass sie alle
    irgendwo möglichen Dinge miteinschließen.
    \begin{itemize}[<+->]
    \item Ja gz, das ist 1. derbe kontraintuitiv und against common
      sense und praxis, das man sehr guten independent reason braucht
      dafür, man stipuliert die Intuition einfach als falsch und
      2. Wie willst Du dann Existenz auszeichnen? HÄ?
    \end{itemize}
  \end{enumerate}
\end{frame}

\begin{frame}{Eine mögliche Lösung: Variable Domain}
  To the Rescue: Gegenstandsbereich der Quantifizierung variiert je
  Welt.
  \begin{itemize}[<+->]
  \item Was heißt das?
  \item Semantik: Interpretation $I$ wie vorher 4-Tupel
    $\langle D, W, R, v \rangle$, nur $v$ wird erweitert so dass
    \begin{itemize}[<+->]
    \item $v(w) \subseteq D$
    \end{itemize}
  \item Wahrheit in einer Interpretation genau wie vorher, nur:
    \begin{itemize}[<+->]
    \item $v_w(\forall x A) = 1$ gdw. jedes $\partial \in v(w)$ Formel
      $A$ erfüllt.
    \item $v_w(\exists x A) = 1$ gdw. mindestens ein
      $\partial \in v(w)$ Formel $A$ erfüllt.
    \end{itemize}
  \item D.h. an unterschiedlichen Welten können Quantoren über
    unterschiedliche Teilmengen von $D$ quantifizieren ($\implies$
    aktualistische Quantifizierung).
  \item Jede Contant Domain Interpretation ist eine spezielle Variable
    Domain Interpretation mit $v(w_0) = v(w_1) = ...$.
  \end{itemize}

\end{frame}

\begin{frame}{Eine mögliche Lösung: Variable Domain}
  Welche Auswirkungen hat das?
  \begin{itemize}[<+->]
  \item `Problem' der klassischen Prädikatenlogik (und der Constant
    Domain): $v(Pa \to \exists x Px) = 1$ Tautologie.
  \item Pegasus ist ein fliegendes Pferd. Also gibt es fliegende
    Pferde.
  \item In der Variable Domain kein Problem:
    \begin{itemize}[<+->]
    \item $v_w(Pa) = 1$ gdw. $v(a) \in v_w(P)$.
    \item Aber $v_w(\exists x Px) = 1$ gdw. mindestens ein Ding
      $\partial$ \textbf{zwei} Bedingungen erfüllt:
      \begin{itemize}
      \item $\partial \in v_w(P)$ und
      \item $\partial \in v(w)$. (Zur Erinnerung: $v(w) \subseteq D$)
      \end{itemize}
    \end{itemize}
  \item $Pa$ kann also gelten an $w$, während $\exists x Px$ an $w$
    nicht gilt!
  \end{itemize}
\end{frame}

\begin{frame}{Variable Domain: Barcan Formula}
  Wie löst das unser Problem mit der Barcan Formula?
  \begin{itemize}[<+->]
  \item \begin{center} BF*:
      $\Diamond \exists x A \to \exists x \Diamond A$
    \end{center}
  \item BF* gilt in Variable Domain nicht mehr! Intuitive Lesart: Wenn
    es an einer Welt der Fall ist, dass es etwas gibt, das $Px$
    erfüllt, dann gibt es etwas an dieser Welt, das an einer anderen
    Welt $Px$ erfüllt.
  \item Gegenbeispiel. Interpretation $I$ mit
    \begin{itemize}[<+->]
    \item $D = \{\partial_a, \partial_b\}$, $W = \{w_0, w_1\}$,
      $R = \{\langle w_0, w_1 \rangle\}$, $v$ eine Fkt. sodass
      \begin{itemize}[<+->]
      \item $v_0(P) = \{\}$, $v_1(P) = \{\partial_b\}$,
      \item $v(w_0) = \{\partial_a\}$, $v(w_1) = \{\partial_b\}$.
      \end{itemize}
    \end{itemize}
  \end{itemize}

\pause
\begin{center}
      \includegraphics[height=2cm]{happydog.png}
\end{center}
\end{frame}

\begin{frame}{Ein Argument gegen Existenz eines Objekts an mehreren Welten}
Also alles gut? Enter Problem der \emph{transworld identity}.
\pause
\begin{itemize}[<+->]
\item Wie kann überhaupt ein und dasselbe Objekt in verschiedenen Welten existieren? Es hat ja dann in einer Welt Eigenschaften, die es in einer anderen Welt nicht hat, also simultan eine Eigenschaft $P$ und nicht die Eigenschaft $P$! Das kann für \emph{alle} Eigenschaften gelten, die es hat! Kann es dann \emph{dasselbe} Objekt sein?

\end{itemize}

\end{frame}

\begin{frame}{Lewisian Counterparts}

\begin{itemize}[<+->]
\item Extreme Version: David Lewis' Counterparts-Theorie. Ein Ding existiert in nur einer Welt, und steht mit maximal ähnlichen Dingen an anderen Welten in einer Counterpart-Relation. Modalität wird dann zusätzlich über die Counterpart-Relation erzeugt. Die muss jedoch nicht symmetrisch oder transitiv sein!
\item Resultierende Logik um einiges schwächer, z.B. gilt selbst in S5 dann nicht mehr $\Box p \to \Box\Box p$ u.a.
\end{itemize}

\end{frame}

\begin{frame}{Alles halb so wild..}

Ein Ding hat eine Eigenschaft immer relativ zu einer Welt, so dass..
\begin{itemize}[<+->]
\item Donald Trump in dieser Welt Republikaner ist,
\item ... aber an einer anderen Welt stattdessen Demokrat.
\item Dann hat er in dieser Welt die Eigenschaft, potentiell ein Demokrat zu sein.
\item Kein Widerspruch!
\item Stattdessen vielleicht epistemisches Problem? Wie können wir wissen, welches Objekt in einer anderen Welt dasselbe ist wie in dieser, wenn die Eigenschaften alle anders sind?
\end{itemize}
\end{frame}

\begin{frame}{Weitere Probleme}

Vergleiche:
\begin{itemize}[<+->]
\item Notwendigerweise ist das Verhältnis von Umfang und Durchmesser eines Kreises irrational.
\item Notwendigerweise ist die Anzahl der Planeten gerade.
\item Sind diese Sätze wahr/falsch?
\item Rigid/Non-Rigid Designators
\end{itemize}


\end{frame}

\begin{frame}{References}
  
  \begin{itemize}
  \item Fitting, Melvin und Mendelsohn, Richard L. First Order Modal
    Logic. Kluwer Academic Publishers, 1998.
  \item Garson, James. Modal Logic, Spring 2016 Edition. The Stanford
    Encyclopedia of Philosophy.
  \item Leary, Christopher C. und Kristiansen, Lars. A Friendly
    Introduction to Mathematical Logic. 2nd Edition. Milne Library,
    2015.
  \item Priest, Graham. Introduction to Non-Classical Logic - From If
    to Is. 2nd ed. Cambridge University Press, 2008.

  \end{itemize}

\end{frame}

\end{document}
